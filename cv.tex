\documentclass[10pt, letterpaper]{article}

% Packages:
\usepackage[
    ignoreheadfoot, % set margins without considering header and footer (test)
    top=2 cm, % seperation between body and page edge from the top
    bottom=2 cm, % seperation between body and page edge from the bottom
    left=2 cm, % seperation between body and page edge from the left
    right=2 cm, % seperation between body and page edge from the right
    footskip=1.0 cm, % seperation between body and footer
    % showframe % for debugging 
]{geometry} % for adjusting page geometry
\usepackage{titlesec} % for customizing section titles
\usepackage{tabularx} % for making tables with fixed width columns
\usepackage{array} % tabularx requires this
\usepackage[dvipsnames]{xcolor} % for coloring text
\definecolor{primaryColor}{RGB}{0, 0, 0} % define primary color
\usepackage{enumitem} % for customizing lists
\usepackage{fontawesome5} % for using icons
\usepackage{amsmath} % for math
\usepackage[
    pdftitle={Rizal Setya Perdana's CV},
    pdfauthor={John Doe},
    pdfcreator={LaTeX with RenderCV},
    colorlinks=true,
    urlcolor=primaryColor
]{hyperref} % for links, metadata and bookmarks
\usepackage[pscoord]{eso-pic} % for floating text on the page
\usepackage{calc} % for calculating lengths
\usepackage{bookmark} % for bookmarks
\usepackage{lastpage} % for getting the total number of pages
\usepackage{changepage} % for one column entries (adjustwidth environment)
\usepackage{paracol} % for two and three column entries
\usepackage{ifthen} % for conditional statements
\usepackage{needspace} % for avoiding page brake right after the section title
\usepackage{iftex} % check if engine is pdflatex, xetex or luatex

% Ensure that generate pdf is machine readable/ATS parsable:
\ifPDFTeX
    \input{glyphtounicode}
    \pdfgentounicode=1
    \usepackage[T1]{fontenc}
    \usepackage[utf8]{inputenc}
    \usepackage{lmodern}
\fi

\usepackage{charter}

% Some settings:
\raggedright
\AtBeginEnvironment{adjustwidth}{\partopsep0pt} % remove space before adjustwidth environment
\pagestyle{empty} % no header or footer
\setcounter{secnumdepth}{0} % no section numbering
\setlength{\parindent}{0pt} % no indentation
\setlength{\topskip}{0pt} % no top skip
\setlength{\columnsep}{0.15cm} % set column seperation
\pagenumbering{gobble} % no page numbering

\titleformat{\section}{\needspace{4\baselineskip}\bfseries\large}{}{0pt}{}[\vspace{1pt}\titlerule]

\titlespacing{\section}{
    % left space:
    -1pt
}{
    % top space:
    0.3 cm
}{
    % bottom space:
    0.2 cm
} % section title spacing

\renewcommand\labelitemi{$\vcenter{\hbox{\small$\bullet$}}$} % custom bullet points
\newenvironment{highlights}{
    \begin{itemize}[
        topsep=0.10 cm,
        parsep=0.10 cm,
        partopsep=0pt,
        itemsep=0pt,
        leftmargin=0 cm + 10pt
    ]
}{
    \end{itemize}
} % new environment for highlights


\newenvironment{highlightsforbulletentries}{
    \begin{itemize}[
        topsep=0.10 cm,
        parsep=0.10 cm,
        partopsep=0pt,
        itemsep=0pt,
        leftmargin=10pt
    ]
}{
    \end{itemize}
} % new environment for highlights for bullet entries

\newenvironment{onecolentry}{
    \begin{adjustwidth}{
        0 cm + 0.00001 cm
    }{
        0 cm + 0.00001 cm
    }
}{
    \end{adjustwidth}
} % new environment for one column entries

\newenvironment{twocolentry}[2][]{
    \onecolentry
    \def\secondColumn{#2}
    \setcolumnwidth{\fill, 4.5 cm}
    \begin{paracol}{2}
}{
    \switchcolumn \raggedleft \secondColumn
    \end{paracol}
    \endonecolentry
} % new environment for two column entries

\newenvironment{threecolentry}[3][]{
    \onecolentry
    \def\thirdColumn{#3}
    \setcolumnwidth{, \fill, 4.5 cm}
    \begin{paracol}{3}
    {\raggedright #2} \switchcolumn
}{
    \switchcolumn \raggedleft \thirdColumn
    \end{paracol}
    \endonecolentry
} % new environment for three column entries

\newenvironment{header}{
    \setlength{\topsep}{0pt}\par\kern\topsep\centering\linespread{1.5}
}{
    \par\kern\topsep
} % new environment for the header

\newcommand{\placelastupdatedtext}{% \placetextbox{<horizontal pos>}{<vertical pos>}{<stuff>}
  \AddToShipoutPictureFG*{% Add <stuff> to current page foreground
    \put(
        \LenToUnit{\paperwidth-2 cm-0 cm+0.05cm},
        \LenToUnit{\paperheight-1.0 cm}
    ){\vtop{{\null}\makebox[0pt][c]{
        \small\color{gray}\textit{Last updated in September 2024}\hspace{\widthof{Last updated in September 2024}}
    }}}%
  }%
}%

% save the original href command in a new command:
\let\hrefWithoutArrow\href

% new command for external links:


\begin{document}
    \newcommand{\AND}{\unskip
        \cleaders\copy\ANDbox\hskip\wd\ANDbox
        \ignorespaces
    }
    \newsavebox\ANDbox
    \sbox\ANDbox{$|$}

    \begin{header}
        \fontsize{25 pt}{25 pt}\selectfont Rizal Setya Perdana

        \vspace{5 pt}

        \normalsize
        \mbox{Malang, Indonesia}%
        \kern 5.0 pt%
        \AND%
        \kern 5.0 pt%
        \mbox{\hrefWithoutArrow{mailto:rizalespe@ub.ac.id}{rizalespe@ub.ac.id}}%
        \kern 5.0 pt%
        \AND%
        \kern 5.0 pt%
        \mbox{\hrefWithoutArrow{tel:+6285755705439}{+6285755705439}}%
        \kern 5.0 pt%
        \AND%
        \kern 5.0 pt%
        \mbox{\hrefWithoutArrow{https://rizalespe.github.io/}{rizalespe.github.io}}%
        \kern 5.0 pt%
        \AND%
        \kern 5.0 pt%
        \mbox{\hrefWithoutArrow{https://www.linkedin.com/in/rizal-setya-perdana-b8a80939/}{linkedin.com/in/rizal-setya-perdana-b8a80939}}%
        \kern 5.0 pt%
        \AND%
        \kern 5.0 pt%
        \mbox{\hrefWithoutArrow{https://github.com/rizalespe/}{github.com/rizalespe}}%
    \end{header}

    \vspace{5 pt - 0.3 cm}

    % Section Data Diri
    \section{Data Diri}
    \begin{onecolentry}
        \begin{highlightsforbulletentries}

        \item \textbf{Nama lengkap:} Rizal Setya Perdana, S.Kom., M.Kom., Ph.D.
        \item \textbf{Alamat Email:} rizalespe@ub.ac.id
        \item \textbf{Nomor telepon:} +6285755705439
        \item \textbf{Website Pribadi}: https://rizalespe.github.io/
        \item \textbf{LinkedIn:} https://www.linkedin.com/in/rizal-setya-perdana-b8a80939/

        \end{highlightsforbulletentries}
    \end{onecolentry}
    
    \section{Deskripsi diri}

        \begin{onecolentry}
        Saya adalah seorang dosen di Universitas Brawijaya dengan spesialisasi bidang keilmuan komputer/informatika. Memiliki latar belakang akademik dengan pendidikan terakhir tingkat doktoral dari Toyohashi University of Technology, saya memiliki pengalaman mengajar dan penelitian terkait kecerdasan artifisial dan data. Saya telah mengajar berbagai mata kuliah seperti Machine Learning Operations (MLOps), Deep Learning, dan Big Data Analytics. Saya juga aktif dalam penelitian dan proyek dengan fokus pada machine learning, serta telah mempublikasikan berbagai karya ilmiah di jurnal dan seminar internasional.
        \end{onecolentry}

        


    
    

    \section{Pendidikan}
        \begin{twocolentry}{
            2009 – 2013
        }
            \textbf{Universitas Brawijaya}, Sarjana Komputer (S.Kom) di Teknik Informatika\end{twocolentry}
        \vspace{0.10 cm}
        \begin{onecolentry}
            \begin{highlights}
                \item IPK: 3.84/4.0
                \item \textbf{Judul Skripsi:} Pengkategorian Pesan Singkat Berbahasa Indonesia pada Jejaring Sosial Twitter dengan Metode Klasifikasi Naïve Bayes
                \item \textbf{Dosen Pembimbing:} Suprapto, S.T dan Rekyan Regasari Mardi Putri, S.T., M.T.M.T, 
            \end{highlights}
        \end{onecolentry}
        \vspace{0.5 cm}
        \begin{twocolentry}{
            2013 – 2015
        }
            \textbf{Institut Teknologi Sepuluh Nopember (ITS)}, Magister Komputer (M.Kom) di Teknik Informatika\end{twocolentry}
        \vspace{0.10 cm}
        \begin{onecolentry}
            \begin{highlights}
                \item IPK: 3.67/4.0
                \item \textbf{Judul Tesis:} Pemilihan Kata Kunci Until Deteksi Kejadian Trivial Menggunakan Autocorrelation Wavelet Coefficients Pada Peringkasan Dokumen Twitter \href{https://repository.its.ac.id/63337/}{(Buku Tesis)}
                \item \textbf{Dosen Pembimbing:} Prof. Dr. Eng. Chastine Fatichah, S.Kom., M.Kom. dan Prof. Dr. Diana Purwitasari, S.Kom., M.Sc. 
            \end{highlights}
        \end{onecolentry}
        \vspace{0.5 cm}
        \begin{twocolentry}{
            2018 – 2022
        }
            \textbf{Toyohashi University of Technology}, Doctor of Philosophy in Engineering (Ph.D.) di Computer Science and Engineering\end{twocolentry}
        \vspace{0.10 cm}
        \begin{onecolentry}
            \begin{highlights}
                \item IPK: 3.91/4.0
                \item \textbf{Judul Disertasi:} Contextualization of Multimodal Sequence Models on Image to Text in Story Generation \href{https://repo.lib.tut.ac.jp/records/2227}{(Buku Disertasi)} 
                \item \textbf{Dosen Pembimbing:} Prof. Yoshiteru Ishida 
            \end{highlights}
        \end{onecolentry}
        


    \section{Pengalaman Kerja}
         \begin{twocolentry}{
            September – Oktober 2024
        }
            \textbf{Peneliti}, Hiroshima University, Peneliti di Laboratorium Learning Engineering Graduate School of Advance Science and Engineering\end{twocolentry}
        \vspace{0.3 cm}
        \begin{twocolentry}{
            November 2017 – Sekarang
        }
            \textbf{Dosen}, Teknik Informatika, Fakultas Ilmu Komputer, Universitas Brawijaya\end{twocolentry}
        \vspace{0.3 cm}
        \begin{twocolentry}{
            2016-2018
        }
            \textbf{Koordinator Bidang Aplikasi Divisi Sistem dan Aplikasi}, Unit Teknologi Informasi dan Komunikasi, Universitas Brawijaya\end{twocolentry}
        \vspace{0.3 cm}
        \begin{twocolentry}{
            2013-2016
        }
            \textbf{Senior Web Programmer}, Unit Teknologi Informasi dan Komunikasi, Universitas Brawijaya\end{twocolentry}
        \vspace{0.3 cm}
        \begin{twocolentry}{
            2011-2013
        }
            \textbf{Junior Web Programmer}, Unit Teknologi Informasi dan Komunikasi, Universitas Brawijaya\end{twocolentry}


\section{Riwayat Mengajar}
    \begin{onecolentry}
        \begin{highlightsforbulletentries}

        \item Machine Learning Operations (MLOps)
        \item Deep Learning
        \item Pengantar Pembelajaran Mesin (Machine Learning)
        \item Pengantar Big Data
        \item Big Data Analytics
        \item Data Mining
        \item Text Mining
        \item Pengantar Sains Data
        \item Pengantar Artificial Intelligence
        \item Data Engineering
        \item Pengantar Sistem Pintar
        \item Penerapan Kecerdasan Artifisial
        \item Teknologi Berbasis Cloud
        \item Keamanan Siber: Teori Teknologi Informasi berbasis AI
        \item Isu Terkini Masalah Gizi, Pangan, dan Dietetik

        \end{highlightsforbulletentries}
    \end{onecolentry}




\section{Publikasi}
        \begin{onecolentry}
            \textit{\textbf{Jurnal:}}
            \vspace{0.50 cm}
        \begin{samepage}
            \begin{twocolentry}{2024}
                \textbf{Prediction of On-time Student Graduation with Deep Learning Method}
            \end{twocolentry}
            \vspace{0.10 cm}
            \begin{onecolentry}
                \mbox{Nathanael Victor Darenoh},
                \mbox{Fitra Abdurrachman Bachtiar},
                 \mbox{\textbf{\textit{Rizal Setya Perdana}}},
                 
                \vspace{0.10 cm}
                \href{https://doi.org/10.5614/itbj.ict.res.appl.2023.18.1.1}{10.5614/itbj.ict.res.appl.2023.18.1.1}
            \end{onecolentry}
        \end{samepage}

        \vspace{0.3 cm}
        \begin{samepage}
            \begin{twocolentry}{2023}
                \textbf{Multilabel Classification for Keyword Determination of Scientific Articles}
            \end{twocolentry}
            \vspace{0.10 cm}
            \begin{onecolentry}
                \mbox{Sulthan Rafif},
                \mbox{\textbf{\textit{Rizal Setya Perdana}}},
                \mbox{Putra Pandu Adikara}
                 
                \vspace{0.10 cm}
                \href{https://doi.org/10.25126/jitecs.202382560}{10.25126/jitecs.202382560}
            \end{onecolentry}
        \end{samepage}

        
        \vspace{0.3 cm}
        \begin{samepage}
            \begin{twocolentry}{2023}
                \textbf{Multilabel Classification for Keyword Determination of Scientific Articles}
            \end{twocolentry}
            \vspace{0.10 cm}
            \begin{onecolentry}
                \mbox{Sulthan Rafif},
                \mbox{\textbf{\textit{Rizal Setya Perdana}}},
                \mbox{Putra Pandu Adikara}
                 
                \vspace{0.10 cm}
                \href{https://doi.org/10.25126/jitecs.202382560}{10.25126/jitecs.202382560}
            \end{onecolentry}
        \end{samepage}

        
        \vspace{0.3 cm}
        \begin{samepage}
            \begin{twocolentry}{2022}
                \textbf{Contextualized Language Generation on Visual-to-Language Storytelling}
            \end{twocolentry}
            \vspace{0.10 cm}
            \begin{onecolentry}
                 \mbox{\textbf{\textit{Rizal Setya Perdana}}},
                 \mbox{Yoshiteru Ishida},
                \vspace{0.10 cm}
                
                \href{https://doi.org/10.1587/transinf.2021KBP0002}{10.1587/transinf.2021KBP0002}
            \end{onecolentry}
        \end{samepage}
        \vspace{0.3 cm}
        \begin{samepage}
            \begin{twocolentry}{2021}
                \textbf{Vision-text time series correlation for visual-to-language story generation}
            \end{twocolentry}
            \vspace{0.10 cm}
            \begin{onecolentry}
                 \mbox{\textbf{\textit{Rizal Setya Perdana}}},
                 \mbox{Yoshiteru Ishida},
                \vspace{0.10 cm}
                
                \href{https://doi.org/10.1587/transinf.2020EDP7131}{10.1587/transinf.2020EDP7131}
            \end{onecolentry}
        \end{samepage}

        \vspace{0.3 cm}
        \begin{samepage}
            \begin{twocolentry}{2018}
                \textbf{Combining Likes-Retweet Analysis and Naive Bayes Classifier within Twitter for Sentiment Analysis}
            \end{twocolentry}
            \vspace{0.10 cm}
            \begin{onecolentry}
                \mbox{\textbf{\textit{Rizal Setya Perdana}}}, 
                \mbox{Aryo Pinandito}
                
                 
                \vspace{0.10 cm}
                \href{https://jtec.utem.edu.my/jtec/article/view/3732}{https://jtec.utem.edu.my/jtec/article/view/3732}
            \end{onecolentry}
        \end{samepage}

        \vspace{0.3 cm}
        \begin{samepage}
            \begin{twocolentry}{2018}
                \textbf{Framework Design for Map-Based Navigation in Google Android Platform}
            \end{twocolentry}
            \vspace{0.10 cm}
            \begin{onecolentry}
                \mbox{Aryo Pinandito},
                \mbox{Agi Putra Kharisma},
                \mbox{\textbf{\textit{Rizal Setya Perdana}}}, 
                 
                \vspace{0.10 cm}
                \href{https://jtec.utem.edu.my/jtec/article/view/3731}{https://jtec.utem.edu.my/jtec/article/view/3731}
            \end{onecolentry}
        \end{samepage}
        
        \vspace{0.3 cm}
        \begin{samepage}
            \begin{twocolentry}{2017}
                \textbf{Framework Design for Modular Web-based Application Using Model-CollectionService-Controller-Presenter (MCCP) Pattern}
            \end{twocolentry}
            \vspace{0.10 cm}
            \begin{onecolentry}
                \mbox{Aryo Pinandito},
                \mbox{Ferdika Bagus Pristiawan Permana},
                \mbox{\textbf{\textit{Rizal Setya Perdana}}}, 
                 
                \vspace{0.10 cm}
                \href{https://doi.org/10.25126/jitecs.20172120}{10.25126/jitecs.20172120}
            \end{onecolentry}
        \end{samepage}
        \vspace{0.3 cm}
        \begin{samepage}
            \begin{twocolentry}{2017}
                \textbf{Klasifikasi Teks Bahasa Indonesia Pada Dokumen Pengaduan Sambat Online Menggunakan Metode K-Nearest Neighbors Dan Chi-square}
            \end{twocolentry}
            \vspace{0.10 cm}
            \begin{onecolentry}
                \mbox{Claudio Fresta Suharno},
                \mbox{M. Ali Fauzi},
                \mbox{\textbf{\textit{Rizal Setya Perdana}}}, 
                 
                \vspace{0.10 cm}
                \href{https://doi.org/10.29080/systemic.v3i1.191}{10.29080/systemic.v3i1.191}
            \end{onecolentry}
        \end{samepage}
        \vspace{0.3 cm}
        \begin{samepage}
            \begin{twocolentry}{2016}
                \textbf{Optimasi Model Segmentasi Citra Metode Fuzzy Divergence Pada Citra Luka Kronis Menggunakan Algoritma Genetika}
            \end{twocolentry}
            \vspace{0.10 cm}
            \begin{onecolentry}
                \mbox{Ghenniy Rachmansyah},
                \mbox{Wayan Firdaus Mahmudy},
                \mbox{\textbf{\textit{Rizal Setya Perdana}}}, 
                 
                \vspace{0.10 cm}
                \href{https://doi.org/10.25126/jtiik.201631163}{10.25126/jtiik.201631163}
            \end{onecolentry}
        \end{samepage}
        \vspace{0.3 cm}
        \begin{samepage}
            \begin{twocolentry}{2015}
                \textbf{Bot spammer detection in twitter using tweet similarity and time interval entropy}
            \end{twocolentry}
            \vspace{0.10 cm}
            \begin{onecolentry}
                 \mbox{\textbf{\textit{Rizal Setya Perdana}}},
                 \mbox{Tri Hadiah Muliawati},
                 \mbox{Reddy Alexandro},
                \vspace{0.10 cm}
                
                \href{https://doi.org/10.21609/jiki.v8i1.280}{10.21609/jiki.v8i1.280}
            \end{onecolentry}
        \end{samepage}
        \vspace{0.3 cm}
        \begin{samepage}
            \begin{twocolentry}{2015}
                \textbf{Pemilihan kata kunci untuk deteksi kejadian trivial pada dokumen Twitter menggunakan Autocorrelation Wavelet Coefficients}
            \end{twocolentry}
            \vspace{0.10 cm}
            \begin{onecolentry}
                 \mbox{\textbf{\textit{Rizal Setya Perdana}}},
                 \mbox{Chastine Fatichah},
                 \mbox{Diana Purwitasari},
                \vspace{0.10 cm}
                
                \href{http://dx.doi.org/10.12962/j24068535.v13i2.a484}{10.12962/j24068535.v13i2.a484}
            \end{onecolentry}
        \end{samepage}
        \vspace{0.3 cm}
        \begin{samepage}
            \begin{twocolentry}{2015}
                \textbf{Prediksi Code Defect Perangkat Lunak Dengan Metode Association Rule Mining dan Cumulative Support Thresholds}
            \end{twocolentry}
            \vspace{0.10 cm}
            \begin{onecolentry}
                \mbox{\textbf{\textit{Rizal Setya Perdana}}}
                 \mbox{Umi Laili Yuhana},
                 
                \vspace{0.10 cm}
                \href{https://doi.org/10.24002/jbi.v6i2.408}{10.24002/jbi.v6i2.408}
            \end{onecolentry}
        \end{samepage}
        \vspace{0.3 cm}
        \begin{samepage}
            \begin{twocolentry}{2015}
                \textbf{Identifikasi Sel Darah Merah Bertumpuk Menggunakan Pohon Keputusan Fuzzy Berbasis Gini Index}
            \end{twocolentry}
            \vspace{0.10 cm}
            \begin{onecolentry}
                 \mbox{Eka Prakarsa Mandyartha},
                 \mbox{Muchammad Kurniawan},
                 \mbox{\textbf{\textit{Rizal Setya Perdana}}}
                \vspace{0.10 cm}
                
                \href{https://doi.org/10.24002/jbi.v6i1.398}{10.24002/jbi.v6i1.398}
            \end{onecolentry}
        \end{samepage}

        
        \vspace{0.5 cm}    
            
        \end{onecolentry}
        % <--------->PROSIDING SEMINAR INTERNASIONAL
        \begin{onecolentry}
            \textit{\textbf{Prosiding seminar internasional:}}
        \end{onecolentry}
        \vspace{0.50 cm}
         \begin{samepage}
            \begin{twocolentry}{2023}
                \textbf{Implementation of Domain Adaptation for Keyword Determination of Scientific Articles Based on Multilabel BERT}
            \end{twocolentry}
            \vspace{0.10 cm}
            \begin{onecolentry}
                \mbox{Sulthan Rafif}, \mbox{\textbf{\textit{Rizal Setya Perdana}}}
                \vspace{0.10 cm}
                
                \href{https://doi.org/10.1145/3626641.3626927}{10.1145/3626641.3626927}
            \end{onecolentry}
        \end{samepage}
        \vspace{0.3 cm}
        \begin{samepage}
            \begin{twocolentry}{2023}
                \textbf{AVIG: A Real-Time Visual Inspection for Guava Grading System Using Computer Vision and XGBoost}
            \end{twocolentry}
            \vspace{0.10 cm}
            \begin{onecolentry}
                \mbox{Raden Arief Setyawan}, \mbox{\textbf{\textit{Rizal Setya Perdana}}},
                \mbox{Made Wena Harilegawa},
                \mbox{Alan Stevrie Balantimuhe},
                \mbox{Achmad Basuki}
                \vspace{0.10 cm}
                
                \href{https://doi.org/10.1145/3626641.3627211}{10.1145/3626641.3627211}
            \end{onecolentry}
        \end{samepage}
        \vspace{0.3 cm}
        \begin{samepage}
            \begin{twocolentry}{2022}
                \textbf{COVID-19 Vaccines in Indonesia: A Public Opinion Study Based on Twitter Data Multidimensional Sentiment Analysis}
            \end{twocolentry}
            \vspace{0.10 cm}
            \begin{onecolentry}
                \mbox{Irvan Wahyu Nurdian},
                \mbox{Dinda Ockta Nooryawati},
                \mbox{Ivan Yulfrian},
                \mbox{Bianca Pingkan Nevista},
                \mbox{Novanto Yudistira},
                \mbox{\textbf{\textit{Rizal Setya Perdana}}},
                \vspace{0.10 cm}
                
                \href{https://doi.org/10.1145/3568231.3568246}{10.1145/3568231.3568246}
            \end{onecolentry}
        \end{samepage}
        \vspace{0.3 cm}
        \begin{samepage}
            \begin{twocolentry}{2022}
                \textbf{Knowledge-Enriched Domain Specific Chatbot on Low-resource Language}
            \end{twocolentry}
            \vspace{0.10 cm}
            \begin{onecolentry}
                \mbox{\textbf{\textit{Rizal Setya Perdana}}},
                \mbox{Putra Pandu Adikara},
                \mbox{Indriati},
                \mbox{Diva Kurnianingtyas},
                \vspace{0.10 cm}
                
                \href{https://doi.org/10.1109/EECCIS54468.2022.9902930}{10.1109/EECCIS54468.2022.9902930}
            \end{onecolentry}
        \end{samepage}
        \vspace{0.3 cm}
        \begin{samepage}
            \begin{twocolentry}{2019}
                \textbf{Pairwise cluster similarity domain adaptation for multimodal deep learning architecture}
            \end{twocolentry}
            \vspace{0.10 cm}
            \begin{onecolentry}
                \mbox{\textbf{\textit{Rizal Setya Perdana}}},
                \mbox{Yoshiteru Ishida},
                \vspace{0.10 cm}
                
                \href{https://doi.org/10.1145/3322645.3322685}{10.1145/3322645.3322685}
            \end{onecolentry}
        \end{samepage}
        \vspace{0.3 cm}
        \begin{samepage}
            \begin{twocolentry}{2019}
                \textbf{Instance-based Deep Transfer Learning on Cross-domain Image Captioning}
            \end{twocolentry}
            \vspace{0.10 cm}
            \begin{onecolentry}
                \mbox{\textbf{\textit{Rizal Setya Perdana}}},
                \mbox{Yoshiteru Ishida},
                \vspace{0.10 cm}
                
                \href{https://doi.org/10.1109/ELECSYM.2019.8901660}{10.1109/ELECSYM.2019.8901660}
            \end{onecolentry}
        \end{samepage}
        \vspace{0.3 cm}
        \begin{samepage}
            \begin{twocolentry}{2017}
                \textbf{Spam detection framework for Android Twitter application using Nave Bayes and K-Nearest Neighbor classifiers}
            \end{twocolentry}
            \vspace{0.10 cm}
            \begin{onecolentry}
                \mbox{Aryo Pinandito},
                \mbox{\textbf{\textit{Rizal Setya Perdana}}},
                \mbox{Mochamad Chandra Saputra},
                \mbox{Hanifah Muslimah Az-zahra},
                \vspace{0.10 cm}
                
                \href{https://doi.org/10.1145/3056662.3056704}{10.1145/3056662.3056704}
            \end{onecolentry}
        \end{samepage}
        \vspace{0.3 cm}
        \begin{samepage}
            \begin{twocolentry}{2016}
                \textbf{RouteBoxer Library for Google Android Platform}
            \end{twocolentry}
            \vspace{0.10 cm}
            \begin{onecolentry}
                \mbox{Aryo Pinandito},
                \mbox{Mochamad Chandra Saputra},
                \mbox{\textbf{\textit{Rizal Setya Perdana}}}
                \vspace{0.10 cm}
                
                \href{https://doi.org/10.1109/APWiMob.2016.7811440}{10.1109/APWiMob.2016.7811440}
            \end{onecolentry}
        \end{samepage}

    \section{Penelitian}
        \begin{twocolentry}{2024/2025}
        Pengembangan Arsitektur Topic Modeling Kepakaran dan Penelitian Dosen dengan Pemanfaatan Large Language Model (LLM) (Studi Kasus: Judul Publikasi Artikel Ilmiah Dosen di Universitas Brawijaya)
        \end{twocolentry}
        \vspace{0.3 cm}
        \begin{twocolentry}{2024/2025}
            Deteksi Kejadian Otomatis di Twitter/X Menggunakan Graph Convolutional Network (GCN) untuk Meningkatkan Ketertarkaitan Konteks Informasi (Studi Kasus: Konten Media Sosial Terkait Universitas Brawijaya)
        \end{twocolentry}
        \vspace{0.3 cm}
        \begin{twocolentry}{2023/2024}
            UB-Chatbot: Transformasi Digital Layanan Percakapan Otomatis Menggunakan Multi-task Deep Learning (Studi Kasus: Helpdesk TIK Universitas Brawijaya)
        \end{twocolentry}
        \vspace{0.3 cm}
        \begin{twocolentry}{2023/2024}
            Optimasi Metode Deep Learning untuk Biomarker Prediksi Potensi Risiko Hipertensi Berdasarkan Varian Genetika Berbasis Mobile
        \end{twocolentry}
        \vspace{0.3 cm}
        \begin{twocolentry}{2023/2024}
            Klasifikasi Multi Label Untuk Penentuan Kata Kunci Artikel Ilmiah
        \end{twocolentry}
        \vspace{0.3 cm}
        \begin{twocolentry}{2023/2024}
            Pengenalan Entitas Bernama untuk Chatbot Berbahasa Indonesia dengan Menggunakan Metode Bi-directional Long Short Term Memory (BI-LSTM)
        \end{twocolentry}
        \vspace{0.3 cm}
        \begin{twocolentry}{2023/2024}
            Implementasi Domain Adaptation untuk Penentuan Kata Kunci Artikel Ilmiah Berbasis BERT Multilabel
        \end{twocolentry}
        \vspace{0.3 cm}
        \begin{twocolentry}{2023/2024}
            Analisis Kinerja Augmented Reality Hypertext Markup Language (ARHTML) dengan Pemanfaatan Web Graphics Library Dan OpenGL Shading Language untuk Pengembangan 3D
        \end{twocolentry}
        \vspace{0.3 cm}
        \begin{twocolentry}{2018/2019}
            Implementasi Fitur Bentuk dan Improved Probabilistic Neural Network untuk Pengenalan Jenis Nilam
        \end{twocolentry}
        \vspace{0.3 cm}
        \begin{twocolentry}{2017/2018}
            Combining Likes-Retweet Analysis and Naive Bayes Classifier Within Twitter for Sentiment Analysis
        \end{twocolentry}
        \vspace{0.3 cm}
        \begin{twocolentry}{2016/2017}
            Deteksi Kejadian Tidak Penting Menggunakan Autocorrelation Wavelet Coefficients pada Peringkasan Dokumen Twitter dengan Pemilihan Kata Kunci
        \end{twocolentry}
        \vspace{0.3 cm}
        \begin{twocolentry}{2016/2017}
            RouterBoxer Library for Google Android Platform
        \end{twocolentry}
        \vspace{0.3 cm}

    \section{Paten/HKI}
        \begin{twocolentry}{2024}
            Modul Perkuliahan Mata Kuliah Text Mining
        \end{twocolentry}
        \vspace{0.3 cm}
        \begin{twocolentry}{2022}
            Chatbot FAQ UB V1.0
        \end{twocolentry}
        \vspace{0.3 cm}
        
    \section{Pengabdian}
        \begin{twocolentry}{2024}
            Pembangunan Sistem Informasi Profil Kelompok Ternak Desa Dawuhan Kec Poncokusumo Kabupaten Malang
        \end{twocolentry}
        \vspace{0.3 cm}
        \begin{twocolentry}{2023}
            Pengembangan Video Profil Sekolah Dasar Sebagai Sarana Teknologi Media Informasi untuk Meningkatkan Promosi Sekolah
        \end{twocolentry}
        \vspace{0.3 cm}
        \begin{twocolentry}{2022}
            Pengembangan Web Responsif UMKM Sebagai Upaya Peningkatan Layanan Pemerintahan Desa Ngijo, Karangploso
        \end{twocolentry}
        \vspace{0.3 cm}
        \begin{twocolentry}{2018}
            Peningkatan Kreatifitas ABM (Anak Buruh Migran) Berbasis Teknologi lnformasi
        \end{twocolentry}
        \vspace{0.3 cm}
        \begin{twocolentry}{2017}
            Sistem Antrian dan Pesan Kamar Online TRAVELOHEALTH untuk Pasien Berbasis Android di Rumah Sakit Kota Malang
        \end{twocolentry}
        \vspace{0.3 cm}
        \begin{twocolentry}{2016}
            Pelatihan Pembuatan Website untuk murid-murid Panti Asuhan Yayasan Taslimiyah dan Yayasan Pendidikan Mu'alimin Nahdatul Ulama Malang
        \end{twocolentry}
        \vspace{0.3 cm}

    \section{Proyek}
        \begin{twocolentry}{
            \href{https://expertlink.id/}{expertlink.id}
        }
            \textbf{ExpertLink}\end{twocolentry}

        \vspace{0.10 cm}
        \begin{onecolentry}
            \begin{highlights}
                \item Expert Link adalah platform yang dirancang untuk menghubungkan peneliti di Universitas Brawijaya dengan berbagai pihak di industri, serta memfasilitasi kolaborasi antarpeneliti di dalam universitas.

                \item Dengan Expert Link, penelitian-penelitian di kampus dikelompokkan ke dalam kategori-kategori yang memudahkan siapa saja menemukan mitra kolaborasi sesuai bidang atau topik yang relevan.

                \item Platform ini diharapkan bisa menjadi wadah untuk menciptakan hubungan yang produktif antara dunia akademis dan industri, membuka peluang kolaborasi yang lebih luas, dan mempercepat penerapan hasil riset di dunia nyata.

            \end{highlights}
        \end{onecolentry}

        \vspace{0.2 cm}

        
        \begin{twocolentry}{
            \href{https://imigrasi.kbritokyo.jp/}{imigrasi.kbritokyo.jp}
        }
            \textbf{Sistem reservasi online untuk pelayanan keimigrasian.}\end{twocolentry}

        \vspace{0.10 cm}
        \begin{onecolentry}
            \begin{highlights}
                \item Sistem informasi untuk melayani proses layanan keimigrasian di KBRI Tokyo 
            \end{highlights}
        \end{onecolentry}

        \vspace{0.2 cm}


    \section{Sertifikasi}
        \begin{twocolentry}{
            \href{https://learn.nvidia.com/certificates?id=12ff9f3ec3ff400ea10e5421aae1acd9}{Link sertifikat}
        }
            \textbf{
Fundamental of Deep Learning}\end{twocolentry}

        \vspace{0.10 cm}
        \begin{onecolentry}
            \begin{highlights}
                \item NVIDIA
                \item Issued Nov 2022
            \end{highlights}
        \end{onecolentry}

        \vspace{0.2 cm}


        \begin{twocolentry}{
            \href{https://www.credly.com/badges/0cf7a632-2800-411a-b78c-e92e5e074bf1/linked_in_profile}{Link sertifikat}
        }
            \textbf{
AWS Certified Solutions Architect – Associate}\end{twocolentry}

        \vspace{0.10 cm}
        \begin{onecolentry}
            \begin{highlights}
                \item AWS
                \item Issued Sept 2022
            \end{highlights}
        \end{onecolentry}

        \vspace{0.2 cm}


        \begin{twocolentry}{
            \href{https://www.credly.com/badges/166a4f25-1e28-4727-884c-5ef9a1982f07?source=linked_in_profile}{Link sertifikat}
        }
            \textbf{Microsoft Certified: Azure AI Fundamentals}\end{twocolentry}

        \vspace{0.10 cm}
        \begin{onecolentry}
            \begin{highlights}
                \item Microsoft
                \item Issued Sept 2020
            \end{highlights}
        \end{onecolentry}

        \vspace{0.2 cm}


    
        \begin{twocolentry}{
            \href{https://www.cloudskillsboost.google/public_profiles/33eeae7f-8b2d-4f96-ac63-3e13b0294e1f/badges/3072143?utm_medium=social&utm_source=linkedin&utm_campaign=ql-social-share}{Link sertifikat}
        }
            \textbf{
Google Cloud Computing Foundations: Cloud Computing Fundamentals}\end{twocolentry}

        \vspace{0.10 cm}
        \begin{onecolentry}
            \begin{highlights}
                \item Google Cloud Skills Boost
                \item Issued Dec 2022
            \end{highlights}
        \end{onecolentry}

        \vspace{0.2 cm}

        

    

    \section{Technologies}

        \begin{onecolentry}
            \textbf{Languages/Framework:} Python, PyTorch, Tensorflow
        \end{onecolentry}

        \vspace{0.2 cm}

        \begin{onecolentry}
            \textbf{Technologies:} Google Cloud, AWS Cloud, NVIDIA
        \end{onecolentry}

\end{document}